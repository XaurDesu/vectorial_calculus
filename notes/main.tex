%%%%%%%%%%%%%%%%%%%%%%%%%%%%%%%%%%%%%%%%%
%  A bright and image filled report style, currently set up here for use with ILM report 8600-219.
%  Contains all that is required, glossaries, content management, references and good looks.
%
% The original template (the Legrand Orange Book Template) can be found here --> http://www.latextemplates.com/template/the-legrand-orange-book
% Original author of the Legrand Orange Book Template:
% Mathias Legrand (legrand.mathias@gmail.com) 
%
% Modifications made for ILM specific reporting
% 
%
% License:
% CC BY-NC-SA 3.0 (http://creativecommons.org/licenses/by-nc-sa/3.0/)
%%%%%%%%%%%%%%%%%%%%%%%%%%%%%%%%%%%%%%%%%
 
%%%%%%%%%%%%%%%%%%%%%%%%%%%%%%%%%%%%%%%%%
% How to use this
%
% Upload a file called FrontCover.jpg to become your new front cover - into the Pictures folder
% Upload files called Heading1.jpg, Heading2.jpg etc to become your new chapter headers - into the Pictures folder
% Make sure these images are the right size to fit their locations and use good quality images
%
% Locate the variables below and set your name, title etc.
%
% If you want to change text colour on the front cover, the areas required are commented below
% If you want to modify text and border colours for your chapter headers go into the structure.tex file and replace the name of the colour (set to ) with a new colour name (find and replace ctrl+f will do this for you).
%
% Add all references into references.bib
% Cite these references by using \cite{referenceName}
%
% Commonly used acronyms or industry specific terms should be added to the glossary
% These terms may then be referenced in the text using \gls{termName}
%
% Finally put some answers in there!
%
% 
% Note: This template is set up specifically for ILM reports, it can be modified for other forms of reports
%
%%%%%%%%%%%%%%%%%%%%%%%%%%%%%%%%%%%%%%%%
 
 
%----------------------------------------------------------------------------------------
%	SET THESE VARIABLES!
%----------------------------------------------------------------------------------------

\def\mytitle{ Vectorial Calculus } % Title of the ILM project
\def\ILMCode{Adbridged, it seems.} % Unique code for the ILM project

\def\author{Jaime Andres Torres B.} % Your name.. 
\def\id{github.com/XaurDesu} % Your unique identifier

\def\date{\today } % Today's date 


 
%----------------------------------------------------------------------------------------
%	PACKAGES AND OTHER DOCUMENT CONFIGURATIONS
%----------------------------------------------------------------------------------------

\documentclass[11pt,fleqn]{book} % Default font size and left-justified equations

\usepackage[dvipsnames]{xcolor}

\input{structure} % Insert the commands.tex file which contains the majority of the structure behind the template

\makeglossaries

%--------------------------------------------------------------------------

% Glossary entries

%--------------------------------------------------------------------------
\newglossaryentry{ETN}
{
    name = {Example Term Name (ETN)},
    description = {What does this term mean? Any examples of it? Further reading? References?}
}

%--------------------------------------------------------------------------

% Document begins here

%--------------------------------------------------------------------------

\begin{document}
\renewcommand{\bibname}{References} % Adds in the link to your references


%----------------------------------------------------------------------------------------
%	TITLE PAGE
%----------------------------------------------------------------------------------------

\begingroup
\thispagestyle{empty}
\AddToShipoutPicture*{\put(0,0){\includegraphics{Pictures/FrontCover.jpg}}} % Image background
\centering
\vspace*{11.3cm}
\par\normalfont\fontsize{35}{35}\sffamily\selectfont

\begin{center}
    % List of Latex Colour names here: https://www.overleaf.com/learn/latex/Using_colours_in_LaTeX
    \textbf{\color{Apricot} \mytitle}  % Modify the name of the colour used to suit your image
    
    \textbf{\color{White}(\ILMCode)} % Modify the name of the colour used to suit your image
    
    \color{white}Uniandes\par % Modify the name of the colour used to suit your image
    
    \vspace*{0.5cm}
    \color{Apricot}\author % Modify the name of the colour used to suit your image
    
    (\id)\par  
\end{center}

\endgroup

%----------------------------------------------------------------------------------------
%	COPYRIGHT PAGE
%----------------------------------------------------------------------------------------


\newpage
~\vfill
\thispagestyle{empty}

\noindent \textbf{Vectorial Calculus - }
\vspace{0.5cm}

\noindent 
\vspace{1cm}

\noindent \textit{First release, \date} % Printing/edition date

%----------------------------------------------------------------------------------------
%	TABLE OF CONTENTS
%----------------------------------------------------------------------------------------

\chapterimage{Heading1.jpg} % Table of contents heading image

\pagestyle{empty} % No headers

\tableofcontents % Print the table of contents itself

%\listoftables %uncomment this if you want to print the list of tables at the start

\pagestyle{fancy} % Print headers again


%----------------------------------------------------------------------------------------
%	Glossary
%----------------------------------------------------------------------------------------

\chapterimage{Heading1.jpg} % Table of contents heading image

\printglossaries



%----------------------------------------------------------------------------------------
%	First set of related questions
%----------------------------------------------------------------------------------------

\chapterimage{Heading2.jpg}
\chapter{Introduction}

\begin{flushright}
    \textit{Go big or go home.}
\end{flushright}

Vectorial calculus is what the title says pretty much, the act of using methods
proper to calculus on vectorial spaces, for the topic of this class generally referring to
merely 3-dimensional ones, at the end of this book you should be able to:

\begin{itemize}
    \item 
\end{itemize}

\vspace{20px}
\vspace{0.5cm} % Adds some vertical whitespace, easier to read

%--------------------------------------------------------------
%	More sections?
%----------------------------------------------------------------------------

\chapter{Linear Algebra Concepts}
\section{Vectors on a three-dimensional space.}

given an $ \mathbb{R}^3 $ space and a point in that space $ P = (a,b,c) $,
we can describe a vector by either connecting the point $ P $ to another point $ Q $, or
by assuming the origin of this space (point $ (0,0,0) $), this is a mathematical object with both a
direction and a magnitude. The direction is given by an angle and the 
magnitude is given by $\sqrt{a_1^2 + a_2^2 + a_3^2}$

As an example, let's assume the vector given by $ P = (1,2,1) $:
\begin{center}
    \includegraphics[scale=0.35]{vector_1.png}

    \textit{Vector formed by $ P = (1,2,1) $}
\end{center}

For this vector, we can calculate the magnitude by replacing the vectorial components by 
the magnitudes of the individual directions, resulting in:

%\begin{gather}
 %    \sqrt{1^2 + 2^2 + 1^2} \\
%
 %    \sqrt{1+4+1} \\
  %  
   %  \sqrt{6} \\
%\end{gather}

\textit{Note, in this course we will be mostly only concerned with }$ \mathbb{R}^3 $
\subsection{Addition and Subtraction}

We can take any $ \vec{a} $ and $ \vec{b} $ vectors on the same space and add them to each other
in the form:

\begin{equation}
    \begin{bmatrix}
        a_1 \\
        a_2 \\
        a_3 \\
    \end{bmatrix}
    +
    \begin{bmatrix}
        b_1 \\
        b_2 \\
        b_3 \\
    \end{bmatrix}
    = 
    \begin{pmatrix}
        a_1 + b_1\\
        a_2 + b_2\\
        a_3 + b_3\\
    \end{pmatrix}
\end{equation}

Such form remains in the case we can do subtraction, which is expressed on the equation:

\begin{equation}
    \begin{bmatrix}
        a_1 \\
        a_2 \\
        a_3 \\
    \end{bmatrix}
    -
    \begin{bmatrix}
        b_1 \\
        b_2 \\
        b_3 \\
    \end{bmatrix}
    = 
    \begin{pmatrix}
        a_1 - b_1\\
        a_2 - b_2\\
        a_3 - b_3\\
    \end{pmatrix}
\end{equation}

This kind of operations have certain properties, shown as:
\begin{gather}
    (\alpha + \beta)\vec[v] = \alpha\vec{v} + \beta\vec{v} \\
    \vec{v} * 1 = \vec{v} \\
    \vec{v} * \vec{0} = \vec{0} \\
    \beta \vec{v} = 
    \begin{pmatrix}
        \beta a_1\\
        \beta a_2\\
        \beta a_3\\
    \end{pmatrix}
\end{gather}

Two vectors $ \vec{a} $ and $ \vec{b} $ are equal if and only if:
\begin{equation}
    \begin{cases}
        \vec{a} \exists \mathbb{R}^3 \\
        \vec{b} \exists \mathbb{R}^3
    \end{cases}
    \implies
    \begin{pmatrix}
        a_1 = b_1\\
        a_2 = b_2\\
        a_3 = b_3\\
    \end{pmatrix}
    \text{
        \textit{note: this can be generalized to 'n' dimensions larger than 0}}
\end{equation}

in either case,  $ \vec{0} $ is the identity of the operation, therefore:
\begin{equation}
    \vec{a} + \vec{0} = \vec{a}
\end{equation}


\subsection{Bases}

A base in $ R^n $ can be found though $ n $ vectors on that plane, such as it 
would happen in $ R^2 $ with:

\begin{equation}
    \lambda \vec{u} + \mu \vec{v}| \lambda , \mu \exists \mathbb{R}
\end{equation}

this equation will form a parallelogram that can express the distorsion of space when
compared to a reference system, which generally is the canonical base formed by the identity.

\subsection{Dot product}
Assume two equal-length vectors of the sort:
\begin{equation}
    \begin{cases}
       \vec{a} = (a_i * n | n \exists \mathbb{R}); |\vec{a}| \exists \mathbb{R}  \\
       \vec{b} = (b_i * n | n \exists \mathbb{R}); |\vec{b}| \exists \mathbb{R} 
    \end{cases}
\end{equation}

in case we wanted to do obtain a scalar number, that corresponded to the sum of the internal products
we could obtain:

\begin{equation}
    A \centerdot B = ||\vec{A}|| ||\vec{B}||  \cos \theta
\end{equation}

for cartesian vectors, we can write this as:
\begin{gather}
    \vec{a} \centerdot \vec{b} = 
    \begin{pmatrix}
        a_1 \\
        a_2 \\
        a_3
    \end{pmatrix}
    \centerdot
    \begin{pmatrix}
        b_1 \\
        b_2 \\
        b_3
    \end{pmatrix}
    = a_1 b_1 + a_2 b_2 + a_3 b_3 \exists \mathbb{R}
\end{gather}


where $ \theta $ is the angle between both vectors. We can get it by calculating

$$ \cos \theta = \frac{\vec{u} x \vec{v}}{||\vec{u}||*||\vec{u}||} $$

If perpendicular, we can assume:

$$ \vec{u} \centerdot \vec{v} = 0 $$

\subsubsection{Notable cases.}

With these rules, we can infere a few interesting cases, which we'll be able to interpolate stuff with.

\paragraph*{Implications}

\begin{itemize}
    \item $ \theta < \frac{\pi}{2} \implies \cos \theta > 0 $
    \item $ \theta > \frac{\pi}{2} \implies \cos \theta < 0 $
    \item $ \theta = \frac{\pi}{2} \implies \cos \theta = 0$
\end{itemize}

\paragraph{Addendum: cosine values}

In vectorial calculus, we'll have certain notable angles that will appear often in exercises.
we can use fractions to get them approximated to numerical values, such values are listed on this table:

\begin{center}
    \begin{tabular}{||c c c||} 
     \hline
     $ \cos 0^o $ & $ \frac{4}{\sqrt{2}} $ & 1\\ [0.5ex] 
     \hline
     $ \cos 30^o $ & $ \frac{3}{\sqrt{2}} $ & ?\\ [0.5ex] 
     \hline
     $ \cos 45^o $ & $ \frac{2}{\sqrt{2}} $ & ?\\ [0.5ex] 
     \hline
     $ \cos 60^o $ & $ \frac{1}{\sqrt{2}} $ & $ \frac{1}{2} $\\ [0.5ex] 
     \hline
     $ \cos 90^o $ & $ \frac{0}{\sqrt{2}} $ & 0\\ [0.5ex] 
     \hline
    
    \end{tabular}
    \end{center}

\paragraph*{Addendum 2: Triangular inequality}

The triangular inequality affirms that:

$ ||\vec{u}+\vec{v}|| \le ||\vec{u}||+||\vec{v}|| $

\subsection{Cross product}

A cross product is, much like the dot product, an operation that seeks to multiply the values between
two vectors. it can be annotated as:

\begin{gather}
    \vec{u} \text{x} \vec{v} = 
    \begin{pmatrix}
        u_1 \\
        u_2 \\
        u_3 \\
    \end{pmatrix}
    *
    \begin{pmatrix}
        v_1\\
        v_2\\
        v_3\\ 
    \end{pmatrix}
    =
    \begin{pmatrix}
        u_2 v_3 - u_3 v_2\\
        u_3 v_1 - v_1 u_3\\
        u_1 v_2 - u_2 v_1 \\ 
    \end{pmatrix}
\end{gather}

This is a non-commutative operation, changing the order of signs will cause the signs to invert,
seen mathematically as:

$$ \vec{u} x \vec{v} = -(\vec{v} x \vec{u}) $$

The vector that the cross product produces is perpendicular to both evaluated vectors.


we can also use the norm of this cross product as a way to calculate the area of the paralleleipied form 
triangulated by $\vec{u}$ and $\vec{v}$ as $$ A = ||\vec{u}x \vec{v}|| $$ 
This can also work for three-dimensional paralleleipied in the following formula:

\begin{equation}
    |\vec{w} \centerdot (\vec{u}x\vec{v})| = ||\vec{w}|| * ||\vec{u}x\vec{v}|| * |\cos \vartheta| 
\end{equation}

we can also say, from this:


\begin{equation}
    \vec{w} \centerdot (\vec{u}x\vec{v}) = \vec{u} \centerdot (\vec{v}x\vec{w}) = \vec{v} \centerdot (\vec{w}x\vec{u})
\end{equation}

\subsection{Determinants}

a determinant is defined as:

\begin{equation}
    det
    \begin{pmatrix}
        a & b \\
        c & d \\
    \end{pmatrix}
\end{equation}

\section{Describing objects in a space.}

\subsection{Lines}

A line is a geometrical object of the form:

\begin{equation}
    r(t) = t \vec{v} + P, t \exists \mathbb{R}
\end{equation}

generating it requires a point and a vector. Point defined by P, and vector defined by an offset 't' and a vector '$\vec{v}$'

\textbf{Example}
\textit{Find the equation of a line 'l' that crosses $A=(2,1,1)$ and $B=(3,5,7)$}

for this, we'll establish the following formula:
\begin{gather}
    l(t) =
    \begin{pmatrix}
        A_1\\
        A_2\\
        A_3\\
    \end{pmatrix}
    +
    t 
    \begin{pmatrix}
        B_1-A_1\\
        B_2-A_2\\
        B_3-A_3\\
    \end{pmatrix}
\end{gather}

Instanced, for this specific case, as:
\begin{gather}
    l(t) =
    \begin{pmatrix}
        2\\
        1\\
        1\\
    \end{pmatrix}
    +
    t 
    \begin{pmatrix}
        3-2\\
        5-1\\
        7-1\\
    \end{pmatrix}
    \\
    l(t) =
    \begin{pmatrix}
        2\\
        1\\
        1\\
    \end{pmatrix}
    +
    t 
    \begin{pmatrix}
        1\\
        4\\
        6\\
    \end{pmatrix}
\end{gather}

Answer is equation 2.15, this can be later expanded into a parametric or simetric form of this line. But before we do that, let's try expanding the reason this works:

\textbf{Example 2}
\textit{Find the equation of the line that joins points $P=(1,2,1)$ and
$ Q=(-1,3,4) $}

we can find the line that joins two points by subtracting the vectors that join them, let's take a look at the 
cartesian plane where we indicate 'P' and 'Q':

\subsection{Vector Projection}

A vector can be projected through the equation:

\begin{equation}
    \frac{\vec{u} \centerdot \vec{v}}{||\vec{v}||^2} \vec{v}
\end{equation}

\subsection{Euclidian Planes}

A plane is the union of all points in a 2-dimensional subset of $ \mathbb{R^3} $ defined by a formula of the type:
\begin{equation}
    i_1 A + i_2 B + i_3 C = D = (D \centerdot || \vec{n} ||)
\end{equation}

Where $ \vec{n} $ is also written as:
\begin{equation}
    \vec{n} = \begin{pmatrix}
        a \\
        b \\
        c
    \end{pmatrix}
\end{equation}

It can be determined by 
\begin{itemize}
    \item three points in $ \mathbb{R^3} $
    \item Two vectors and a point in $ \mathbb{R^3} $
    \item a point and the normal vector in $ \mathbb{R^3} $
\end{itemize}

\section{Cylindrical and spherical coordinates.}

When trying to define parts of a line in algebra, we'll usually be looking at
coordinates, be them polar or cartesian. In either case, their information can be 
converted to the other system through the following formulas.

\begin{gather}
    \begin{cases}
        \rho = \sqrt{x^2+y^2} \\
        \theta = \arctan(\frac{y}{x})
    \end{cases}
    \text{\textit{cartesian to polar}}\\
    \begin{cases}
        \alpha_x = \rho \cos \theta \\
        \alpha_y = \rho \sin \theta
    \end{cases}
    \text{\textit{polar to cartesian}}
\end{gather}

Cartesian coordinates generally translate well to other dimensional spaces, such as would be the case
for $ \mathbb{R}^3 $, however, polar coordinates as we know them usually aren't as translatable in a direct
manner, and expressing them in three-dimensional spaces might be better suited to 
be expressed on a cylindrical or spherical condition. 
\subsubsection*{Cylindrical coordinates}
In the case of cylindrical coordinates, the translation is probably the most intuitive, by computing
a cylinder with polar coordinates that indicate an (x,y) position, and a 'Z' variable indicating height 
that allows us to project the vector on a third dimension, this 'z' variable is exactly the same as it would be
on a cartesian model. We can express it like such:

$ \vec{v} = ( \rho, \theta, Z) $

conversion to a cartesian model can be expressed as:

\begin{gather}
    \vec(\alpha) =
    \begin{cases}
        \alpha_x = \rho \cos \theta \\
        \alpha_y = \rho \sin \theta \\
        \alpha_z = Z
    \end{cases}
    \text{\textit{Cylindrical to cartesian}}
\end{gather}

\subsubsection*{Spherical coordinates}

A spherical coordinate is formed by a tuple:

\begin{gather}
    (\rho, \theta ,\phi);
    \begin{cases}
        \rho \geq  0 \\
        0 \le \theta \le 2\pi \\
        0 \le \phi \le \pi    
    \end{cases}
\end{gather}

Where $ \rho $ is the magnitude of the vector, $\theta$ is the (x,y) coordinates, and
$\phi$ is the (y,z) angle. They must adhere to the following for it to be geometrically coherent:

\begin{equation}
    \begin{cases}
        \rho > 0 \\
        0 \le \phi \le \pi \\
        0 \le \theta \le 2\pi
    \end{cases}
\end{equation}

this tuple can generate two vectors:
\begin{equation}
    \begin{cases}
        \rho \sin \phi\\
        \rho
    \end{cases}
\end{equation}

And can be converted to a cartesian model as such:

\begin{gather}
    \vec(\alpha) =
    \begin{cases}
        \alpha_x = \rho \sin \phi \cos \theta \\
        \alpha_y = \rho \sin \phi \sin \theta \\
        \alpha_z = \rho \cos \phi
    \end{cases}
    \text{\textit{Spherical to cartesian}}
\end{gather}

\paragraph{Example}
Imagine the following spherical vector:
$$
\begin{cases}
    \rho = 2 \\
    \theta = \frac{\pi}{2}\\
    \phi = \frac{\pi}{4}
\end{cases}
$$

\textit{How do we convert it to a cartesian vector?}

We'll get the vector by simply replacing the previous formulas with the 
values provided as it follows:

\begin{equation}
    x = 2 \sin(\frac{\pi}{4}) \cos(\frac{\pi}{2}) \\
    y = 2 \sin(\frac{\pi}{4}) \sin(\frac{\pi}{2}) \\
    z = 2 \cos(\frac{\pi}{4})
\end{equation}

We can also invert this equation and get a cartesian vector to its spherical form
through the following formula:
\begin{gather}
    \begin{cases}
        \rho = \sqrt{x^2 + y^2 + z^2} \\
        \theta = \arctan{\frac{y}{x}} \\
        \phi = \arccos{\frac{z}{\sqrt{x^2 + y^2 + z^2}}}
    \end{cases}
    \text{\textit{Cartesian to Spherical}}
\end{gather}

Such a model works, as we might imagine, like a sphere. where we express the possible vectors
through a sphere of $ \rho $ radius.

\section{n-dimensional Euclidian Spaces}

in an n-dimensional euclidian space, we can determine:

\begin{equation}
    \mathbb{R}^n, \vec{x} \begin{pmatrix}
        x_1
        \dots
        x_n
    \end{pmatrix}
    ; \mathbb{C}
    \text{\textit{ Operations:}}
    \begin{cases}
        \vec{x}+\vec{y} \\
        \alpha \vec{x} \\
        \vec{x} \centerdot \vec{y} \\
    \end{cases}
\end{equation}


\subsection{Cauchy-Schwartz Inequality}

This inequality determines that the internal product is lesser or equal to the
multiplication of the norms of two vectors, written as:

\begin{gather}
    \text{Let: } \vec{x},\vec{y} \exists \mathbb{R}^n \text{ then: }\\
    |\vec{x}\centerdot\vec{y}| \le ||\vec{x}|| \centerdot ||\vec{y}||
\end{gather}

and it is equal if and only if:

\begin{equation}
    \vec{x} = \lambda \vec{y} \text{ or either } \begin{cases}
        \vec{x} = 0 \\
        \vec{y} = 0
    \end{cases}
\end{equation}

\section{Matrices}

A matrix is a numerical representation of values in $ \mathbb{R}^n $. They can represent 
planes, vectors, or even hyperplanes in $ \mathbb{R}^n | n > 3 $. An example in $\mathbb{R}^2$
would be:

\begin{equation}
    \begin{pmatrix}
        a & b \\
        c & d
    \end{pmatrix}
\end{equation}

Notable matrices include:
\begin{itemize}
    \item Identity: $ \begin{pmatrix}
        1 & \dots & 0 & \dots & 0 \\
        0 & \dots & 1 & \dots & 0 \\
        0 & \dots & 0 & \dots & 1
    \end{pmatrix} $ = Id
\end{itemize}

\subsection{Inverible Matrices}

We can invert a matrix if a $ B_{nxn} $ matrix exists such as:
\begin{equation}
    AB = BA = Id
\end{equation}

we can also use the determinant to check this, as:

\begin{equation}
    det(A) \begin{cases}
        = 0 \text{: is Invertible} \\
        \neq 0 \text{: is not Invertible} \\
    \end{cases}
\end{equation}

\subsection{Matrix multiplication}

We can multiplicate a matrix by another one if we define the multiplication as:

\begin{equation}
    AxB = C_{ij} = \sum_{k = 1}^{n} a_{ik} b_{jk}  
\end{equation}

\paragraph{Example}

We want to multiply two matrices as:

\begin{equation}
    \begin{pmatrix}
        a & b\\
        c & d
    \end{pmatrix}
    x
    \begin{pmatrix}
        1 & 2 \\
        0 & 1
    \end{pmatrix}
\end{equation}

Therefore if we try doing AxB and BxA:

\begin{gather}
    AxB =
    \begin{pmatrix}
        a & 2a + b \\
        c & 2c + d
    \end{pmatrix}\\
    BxA =
    \begin{pmatrix}
        a+2c & b + 2d \\
        c & d
    \end{pmatrix}\\
\end{gather}

As we can see, matrix multiplication is not commutative, but rather it is defined
by the order on which A and B are written


$$ Ae_j = A_j $$

\chapter{Functions}
\section{Geometry of functions with values in $ \mathbb{R} $}

We can affirm that a function is two or three dimensional if, respectively:

\begin{gather}
    y = f(x) \implies \begin{cases}
        (x,y): y = f(x), x \exists D(f)
    \end{cases} \\
    z = f(x,y) \implies \begin{cases} 
    (x,y,z): z = f(x,y), (x,y) \exists D(f)
    \end{cases}
\end{gather}

\pagebreak
\paragraph*{Example:}
\textit{3,1,1: Graphicate the following:}
$$ z = \frac{6-x-2y}{3} $$
\includegraphics[scale = 0.5]{Pictures/plane3,1.png}

As we can see, the graph makes sense because:

\pagebreak
\textit{3,1,2: Graphicate the following:}

$$ z = 1-x = f(x,y) $$
\includegraphics*[scale=0.5]{plane3,1,2.png}
\pagebreak

\textit{3,1,3: Graphicate the following:}

$$ z = x^2 + y^2 $$

\includegraphics[scale = 0.5]{Pictures/plane3,1,3.png}
this is a three-dimensional parabola, comparable to it's two dimensional form, yet 
working on another 

it can guarantee:

$$ \begin{cases}
    x = 0: z = y^2 \\
    y = 0: z = x^2
\end{cases}$$

and this equation can be deduced through this behavior. As:

$$ z = z_0 : x^2 + y^2 = z $$

there are level curves, which project the curve generated by such mathematical artifacts as:


% Simply upload additional images Heading5.jpg, Heading6.jpg etc. into the pictures folder

% \chapterimage{Heading5.jpg}
% \chapter{Third Set of Questions}


%----------------------------------------------------------------------------------------
%	References
%----------------------------------------------------------------------------------------

\chapterimage{Heading4.jpg} % Chapter heading image

\bibliographystyle{plain} % Change this to IEEE or Harvard etc.
\bibliography{references}


\end{document}